\documentclass[a4paper,11pt]{article}
\usepackage[english]{babel}
\usepackage{a4wide}
\usepackage{float}
\usepackage{enumerate}
\usepackage{hyperref}

\title{IN4086 Data Visualization \\
Project Proposal \\ }
\author{
Friso Abcouwer \\
4019873 \\
\and
Marijn Goedegebure \\
4013484
}

\begin{document}
\maketitle{}
\newpage

\section{Problem Description}
In the Netherlands, there are a lot of available methods of public transportation, such as trams, trains and buses.
A lot of people rely on websites like \url{http://9292.nl} to help them navigate the public transportation system, but might not be aware of all of the public transportation options in their area. The people responsible for managing the public transportation network want to know where there is a high demand for new or expanded services, and they need some way to find areas like this.
For these reasons, we decided to visualize public transportation in the Netherlands. While there is open data available for most modes of public transportation, we decided to focus on transportation by bus for our project. In this report, we will explain how we acquired our datasets and how we filtered them, as well as motivate our design choices.  

\section{Design Process}
\subsection{Datasets:}
The website OpenOV~\footnote{http://openov.nl} offers a lot of public transportation datasets. Since the bus stop datasets already had longitude and latitude data, we realised this would make it possible to visualize them geographically. In the end we used the following datasets:
\begin{itemize}
\item Geographical data of the borders of municipalities and provinces (Imergis~\footnote{http://www.imergis.nl/asp/47.asp})
\item Inhabitant counts of municipalities and provinces (CBS~\footnote{www.cbs.nl})
\item Geographical bus stop data (OpenOV)
\item Lines per bus stop data (OpenOV)
\end{itemize}

As can be seen from the list above, we used multiple different datasets. Unfortunately the datasets are not in the same format. To use this data we had to convert some datasets to a more usable format. This required the use of some specific tools. We used QGIS~\footnote{http://www2.qgis.org/} to convert the geographical data of the borders of municipalities and provinces from shapefiles to kml files. Unlike shapefiles, kml files are supported by the Google Maps API and can also be converted to CSV files.
\subsection{Processing the Data:}
We used Java and Javascript to process the data. Since we could not find a dataset linking towns to their municipality and province, we had to assign a municipality and province to each bus stop ourselves. The ProvinceProcessor class outputs a file with this information.
This file and the data from OpenOV and CBS, are used in the DataSetProcessor class to aggregate information into a single dataset that can be used for visualization. It can output a file with information on every single bus stop, or a file with information for every municipality and province.
While working on this conversion, we noticed several types of data in the OpenOV file that were not usable for our purposes. These included places where there is currently no bus route scheduled, badly formatted lines or entries that represented a train station instead of a bus stop (not to be confused with bus stops \textbf{at} train stations, of course). This brought our total down to about 50.000 stops from 55.000, the vast majority being unscheduled entries. Matching stops to provinces went relatively smoothly: less than 250 stops could not be assigned to a province based on the geometrical data. Since this was less than 0,5\% of our data, we decided to omit these entries from our visualization rather than find a method to include them.
\subsection{Presenting the Data:}
To achieve our visualization purposes, that were described in the problem description, we used different types of visualizations: an interactive map, the second one are graphs based on the raw data. We will explain in short why we chose to use each of these visualizations.\\

Firstly, an interactive map lets the user look through the raw data. This comes with a risk of the user seeing too much information and being confused by this. We solved this by making the map adapt to the user's zoom level and selection. For example: A user can select and zoom in on a municipality and can see the individual bus stops in that municipality. When the user zooms out, the individual bus stops are hidden and the municipality pop-up remains.\\
The user can also select whether he or she wants to interact with the data sorted by provinces or municipalities. This also reduces the amount of data the user can see at the same time, further reducing the risk of the user not understanding what he or she is looking at.\\

Secondly, we analysed our datasets and made observations based on choropleth maps, bar charts and pie charts. These can be found on the province and municipality overview pages. These maps and charts aggregate data the user sees into a single graph focusing on a particular aspect. For each graph, we included a brief discussions of how it can be interpreted and what the implications of these interpretations are. These maps and charts give the user a quick overview of the data and present some interesting facts that could spark a further inspection using the interactive map.

\end{document}

