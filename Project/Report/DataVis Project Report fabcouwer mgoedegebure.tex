\documentclass[a4paper,11pt]{article}
\usepackage[english]{babel}
\usepackage{a4wide}
\usepackage{float}
\usepackage{enumerate}
\usepackage{hyperref}

\title{IN4086 Data Visualization \\
Project Proposal \\ }
\author{
Friso Abcouwer \\
4019873 \\
\and
Marijn Goedegebure \\
4013484
}

\begin{document}
\maketitle{}
\newpage

\section{Problem Description}
In the Netherlands, there are a lot of available methods of public transportation, such as trams, trains and buses.
A lot of people rely on websites like \url{http://9292.nl} to help them navigate the public transportation system, but might not be aware of all of the public transportation options in their area. The people responsible for managing the public transportation network want to know where there is a high demand for new or expanded services, and they need some way to find areas like this.
For these reasons, we decided to visualize public transportation in the Netherlands. While there is open data available for most modes of public transportation, we decided to focus on transportation by bus for our project. In this report, we will explain how we acquired our datasets and how we filtered them, as well as motivate our design choices.  

\section{Design Process}
\begin{itemize}
\item Where/how did we find data
\item How did we filter the data and why
\item How did we present the data and why
\end{itemize}

\end{document}

